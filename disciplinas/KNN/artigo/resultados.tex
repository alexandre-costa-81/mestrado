\section{Resultados}

Foram coletados 174 exemplos divididos igualmente entre mãos para cima e para baixo. Para descobrir a quantidade de exemplos necessários para que o KNN possa reconhecer essas posições, foram testadas diferentes quantidades desses exemplos coletados no conjunto de treinamento. Inicialmente 5\%, 10\%, 15\%, então 20\%, 30\% e 40\%. Se observou que 30\%(52 exemplos) já alcança mais de 99\% de taxa de acerto, então se diminuiu para 25\% dos dados para treinamento, ainda com 99\% de acertos. Já com 20\% dos dados para treinamento, o resultado fica em 95\% de acertos com base de teste em cima do resto dos dados. O M5Rules também alcançou 99\% de taxa de acerto, porém com uma quantidade menor que o KNN, cerca de 20\% dos resultados são suficientes, foram testados 5\%, 10\%, 15\% e 20\%. 

\begin{table}[h]
\centering
\caption{Resultados}
\label{my-label}
\begin{tabular}{|l|l|l|}
\hline	
     & KNN & M5Rules \\
\hline	
5\%  & 64.84\% & 28.22\% \\
\hline	
10\% & 91.71\% & 88.18\% \\
\hline	
15\% & 93.91\% & 87.98\% \\
\hline	
20\% & 95.68\% & 99.00\% \\
\hline	
30\% & 100.00\% & 100.00\% \\
\hline	
40\% & 100.00\% & 100.00\% \\
\hline	
\end{tabular}
\end{table}

Já a árvore M5Rules rodando na placa obteve uma taxa de erro de cerca de 3\%, devido a perdas da comunicação serial.
\section{Kinect Vers\~ao 2} \label{sec:kinectversion2}

O Kinect é um sensor equipado de uma câmera RGB, um sensor de profundidade
composto de um emissor de luz infravermelho e uma câmera sensível à
profundidade.

O principio básico por trás do sensor de profundidade do Kinect é a emissão de
um padrão de infravermelho e a captura simultânea da imagem desse infravermelho
com uma câmera tradicional equipada com um filtro, que permite capturar o 
infravermelho e bloquear outras formas de onda. O processador de imagens do 
Kinect usa as posições relativas dos pontos 
no padrão do infravermelho para calcular o deslocamento da profundidade em cada
posição de pixel na imagem.

Cada pixel do mapa de profundidade representa a distância cartesiana, em 
milímetros, do plano da câmera até o objeto mais próximo, naquela coordenada
(x, y) em particular. Se o valor do pixel é 0, isso indica que o sensor não 
encontrou nenhum objeto no seu espaço de alcance naquela localização (x, y).
Essa projeção é mencionado como espaço de profundidade. Os valores correntes de
profundidade são as distâncias do plano da câmera, ao invés das do sensor 
propriamente dito.